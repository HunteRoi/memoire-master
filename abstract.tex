\documentclass{article}
\usepackage{config}

\begin{document}
\maketitle

\begin{abstract}
Ce travail de fin d’études porte sur la conception et le développement d’une interface de programmation accessible destinée à l’initiation à la robotique éducative.
L’objectif principal est de proposer un environnement intuitif qui permette à un public scolaire, du primaire au secondaire, de programmer et de contrôler des robots e-puck2 sans nécessiter de compétences préalables en programmation.

Le projet répond à une problématique éducative : rendre la robotique plus accessible en combinant des approches visuelles (programmation par blocs) et textuelles (Python) dans un même outil.
L’interface a été pensée pour être adaptable à l’utilisateur (âge, langue, ...), avec différents modes d’utilisation : navigation et exploration.
Elle intègre également un firmware dédié, fonctionnant comme serveur, pour gérer la communication temps réel entre l’interface et les robots via WebSocket.

La méthodologie a comporté plusieurs étapes :
\begin{itemize}
    \item \textbf{Analyse des besoins} en robotique éducative et étude comparative des outils existants (Blockly, Scratch, Open Roberta Lab, Webots, etc.).

    \item \textbf{Conception de l’architecture} logicielle et des maquettes d’interface.

    \item \textbf{Développement du client et du firmware}, suivi de l’intégration et des tests.

    \item \textbf{Pré-évaluation utilisateur} auprès d’un panel diversifié afin d’évaluer l’intuitivité et l’ergonomie.
\end{itemize}

Les résultats montrent que l’outil développé répond globalement aux attentes : les utilisateurs novices peuvent réaliser des tâches de programmation robotique simples après une courte prise en main.
Les retours recueillis ont permis de mettre en lumière des pistes d’amélioration, notamment en matière de clarté visuelle, d’optimisation des performances et d’ajout de scénarios pédagogiques plus complexes.

En conclusion, ce projet démontre qu’une approche centrée sur l’accessibilité et la modularité peut considérablement faciliter l’apprentissage de la robotique, tout en offrant une base solide pour de futurs développements dans un cadre éducatif.
\end{abstract}
\end{document}
