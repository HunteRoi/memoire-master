\subsection{Méthodologie} \label{sec:méthodologie}

Ce projet combine des dimensions techniques, pédagogiques et exploratoires et adopte donc une approche Agile, et plus précisément la méthode \textbf{Scrum}.  
Les fonctionnalités visées sont présentées dans la \autoref{sec:annexes_b}.

L'organisation temporelle du projet est illustrée par un diagramme de Gantt complet également disponible en \autoref{sec:annexes_b}, qui couvre l’ensemble du processus de conception et de rédaction.
La figure suivante présente sous forme synthétique les phases de création du prototype technique.

\begin{figure}[H]
    \centering
    \begin{subfigure}{\linewidth}
        \centering
        \resizebox{\linewidth}{!}{
            \begin{ganttchart}[
                hgrid,
                vgrid,
                time slot format=isodate,
                bar height=0.6,
                bar/.style={fill=blue!60},
                bar label font=\small\bfseries,
                milestone/.style={fill=red},
                link/.style={->, thick}
            ]{2025-06-01}{2025-07-10}
                \gantttitlecalendar{month=name, day} \\
                \ganttbar{Conception architecture logicielle}{2025-06-01}{2025-06-22} \\
                \ganttbar{Développement interface utilisateur}{2025-06-23}{2025-07-10} \\
            \end{ganttchart}
        }
        \caption{\label{fig:gantt_phase1} Phase 1 — Conception de l’architecture et démarrage du développement de l’interface.}
    \end{subfigure}

    \vspace{1em}

    \begin{subfigure}{\linewidth}
        \centering
        \resizebox{\linewidth}{!}{
            \begin{ganttchart}[
                hgrid,
                vgrid,
                time slot format=isodate,
                bar height=0.6,
                bar/.style={fill=blue!60},
                bar label font=\small\bfseries,
                milestone/.style={fill=red},
                link/.style={->, thick}
            ]{2025-07-11}{2025-08-07}
                \gantttitlecalendar{month=name, day} \\
                \ganttbar{Développement firmware robot}{2025-07-11}{2025-08-03} \\
                \ganttbar{Intégration client-robot}{2025-07-22}{2025-08-03} \\
                \ganttbar{Tests utilisateur}{2025-08-04}{2025-08-07} \\
            \end{ganttchart}
        }
        \caption{\label{fig:gantt_phase2} Phase 2 — Développement du firmware, intégration et validation par des tests.}
    \end{subfigure}

    \caption{\label{fig:gantt_prototype_phases} Planning de développement du prototype par phases}
\end{figure}