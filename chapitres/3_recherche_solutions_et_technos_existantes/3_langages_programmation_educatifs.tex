\subsection{Langages de programmation éducatifs} \label{sec:lgg_progra_educatifs}

L’apprentissage de la programmation et de la robotique éducative repose sur de nombreux outils et technologies visant à rendre ces disciplines plus accessibles.  
Parmi les plus populaires, on retrouve \textit{Scratch}, \textit{Blockly}, \textit{ThymioSuite} et \textit{Tinkercad}, qui facilitent la compréhension des concepts fondamentaux grâce à des interfaces visuelles intuitives.  

Une tendance marquante dans ce domaine est l’essor des \acrfull{lpv}, en particulier la \textit{programmation par blocs}, souvent utilisée dans des approches \textit{no-code} \autocites{noauthor_software_nodate}.  
Ce paradigme simplifie la représentation des concepts complexes et rend l’apprentissage plus intuitif, notamment pour les débutants \autocites{noauthor_visual_2025}{batni_current_2025}.  

Dans l'enseignement supérieur, en particulier en Belgique, ce paradigme est intégré à des initiatives pédagogiques visant à éveiller l’intérêt des jeunes en favorisant la collaboration et l'apprentissage immersif.  
À titre d’exemple, l’\textit{Initiation ludique à la programmation avec Minecraft} à l’Hénallux \autocite{christophe_leclere_initiation_2020}, et le \textit{Thymio Escape Game} \autocite{christian_giang_thymio_nodate} mis en place à l’UNamur lors du Printemps des Sciences 2025, illustrent cette approche.

\subsubsection{Caractéristiques et limites}
Les outils éducatifs existants répondent à des besoins pédagogiques variés.  
Le tableau \ref{tab:comparison_tools} (en \autoref{sec:annexes_a}) résume leurs principales caractéristiques.

Il est possible de citer CoppeliaSim (V-REP) et Webots dans le domaine de la \textit{simulation robotique avancée}, soit la modélisation de scénarii complexes ; Blockly, Scratch, ThymioSuite et Unreal Engine Blueprints pour la \textit{programmation visuelle}, soit un accès presqu'intuitif à la logique algorithmique et le développement logiciel ; des environnements comme Arduino IDE et Lab Open Roberta pour la \textit{programmation textuelle} ; Tinkercad et Processing dans le domaine de la \textit{modélisation et conception électronique}.

\subsubsection{Conclusion}

L'écosystème des outils éducatifs en programmation et robotique est riche et diversifié, chaque technologie présentant des avantages en fonction du public cible et des objectifs pédagogiques.
Associer plusieurs de ces outils permet de proposer des parcours d’apprentissage progressifs, depuis les bases jusqu’aux projets complexes de robotique et simulation.

Inspirée de la ThymioSuite, l'intégration d’un langage de programmation de haut niveau tel qu'Aseba semble être une approche pertinente.
Toutefois, ce langage a été développé pour interagir avec les fonctionnalités de base du robot et ne prend pas en charge l'ensemble des extensions avancées de l'e-puck2.
Ainsi, certaines fonctionnalités comme l’exploitation des capteurs telle que la caméra 360°, ou l'utilisation du système de contrôle Pi-puck, nécessiteraient l’usage d’un langage plus flexible.

Afin d’assurer une accessibilité optimale et une montée en compétences progressive, une approche hybride semble plus adaptée : 
\begin{enumerate}
    \item \textbf{Du point de vue de l’interface} : une combinaison de programmation par blocs et textuelle offrirait une transition fluide entre l’initiation et des concepts plus avancés.  
    \item \textbf{Du point de vue logiciel} : associer des langages comme Python et C++ permettrait de mieux exploiter les extensions matérielles de l'e-puck2.
\end{enumerate}
Cette approche garantirait un équilibre entre accessibilité et flexibilité, facilitant l’utilisation du robot tant dans un cadre éducatif que pour des applications plus avancées.
