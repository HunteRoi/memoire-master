\subsubsection{Fonctionnalités}

Les fonctionnalités du projet sont organisées en \textbf{Epics}, divisées en \textbf{Features}, elles-mêmes décomposées en \textbf{\acrfull{us}} qui peuvent être ainsi divisées en \textbf{Technical Tasks} lors du développement.
Ce découpage permet d’assurer une approche structurée du développement, en garantissant une intégration fluide et progressive des différentes capacités du robot e-puck2.

Chaque \acrshort{us} respecte l'acronyme \acrfull{invest} \autocite{alliance_what_2015}.
Elles sont également rédigées en tant que "Job Stories" afin de changer la perception des besoins et ainsi remettre au centre ce qui doit être fait pour le client \autocite{klement_replacing_2018}.

\paragraph{Epic 1 - Contrôle des mouvements du robot}

\subparagraph{Feature 1.1 - Navigation en environnement défini}

\textbf{\acrshort{us} 1.1.1} Lorsque je veux que le robot suive un trajet spécifique, je veux pouvoir définir une séquence de mouvements afin qu'il puisse exécuter un parcours précis.

\textbf{\acrshort{us} 1.1.2} Lorsque j’ai besoin d’adapter la vitesse du robot, je veux pouvoir choisir entre plusieurs niveaux de vitesse afin d’ajuster sa rapidité en fonction du scénario.

\subparagraph{Feature 1.2 - Gestion des obstacles et détection de collisions}

\textbf{\acrshort{us} 1.2.1} Lorsque le robot rencontre un obstacle en mode pré-programmé, je veux qu’il s’arrête immédiatement afin d’éviter une collision.

\textbf{\acrshort{us} 1.2.2} Lorsque le robot détecte un obstacle en mode aléatoire, je veux qu’il redirige sa trajectoire automatiquement afin qu’il puisse continuer son déplacement sans interruption.

\paragraph{Epic 2 - Exploration de l’environnement}

\subparagraph{Feature 2.1 - Détection et suivi des zones de couleur}

\textbf{\acrshort{us} 2.1.1} Lorsque le robot évolue sur une surface colorée, je veux qu’il identifie les zones spécifiques en fonction de leur couleur afin de déclencher des actions adaptées.

\textbf{\acrshort{us} 2.1.2} Lorsque le robot détecte une couleur spécifique, je veux pouvoir lui attribuer une action prédéfinie afin qu’il puisse interagir dynamiquement avec son environnement.

\paragraph{Epic 3 - Interaction avec l’utilisateur et retour d’état}

\subparagraph{Feature 3.1 - Affichage des états du robot via \acrshort{led}s et son}

\textbf{\acrshort{us} 3.1.1} Lorsque l’état du robot change, je veux qu’il affiche une couleur spécifique sur ses \acrshort{led}s afin que je puisse identifier rapidement son statut.

\textbf{\acrshort{us} 3.1.2} Lorsque le robot rencontre un événement important, je veux qu’il émette un son spécifique afin que je sois alerté sans avoir à regarder directement le robot.

\paragraph{Epic 4 - Observation et comportements collectifs}

\subparagraph{Feature 4.1 - Coordination entre plusieurs robots}  
Bien qu'en dehors de la portée de ce projet, cette fonctionnalité pourrait être un atout majeur pour le futur de ce projet.

\textbf{\acrshort{us} 4.1.1} Lorsque plusieurs robots interagissent, je veux qu’ils puissent synchroniser leurs actions afin de réaliser des comportements coordonnés.

\textbf{\acrshort{us} 4.1.2} Lorsque deux robots participent à une simulation de poursuite, je veux leur attribuer un rôle de chasseur et de proie afin d’observer un scénario de type "jeu du chat et de la souris".
