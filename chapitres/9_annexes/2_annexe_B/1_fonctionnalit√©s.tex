\subsubsection{Fonctionnalités} \label{sec:features}

Les fonctionnalités du projet sont organisées en \textbf{Epics}, divisées en \textbf{Features}, elles-mêmes décomposées en \textbf{\acrfull{us}} qui peuvent être ainsi divisées en \textbf{Technical Tasks} lors du développement.
Ce découpage permet d’assurer une approche structurée du développement, en garantissant une intégration fluide et progressive des différentes capacités du robot e-puck2.

Chaque \acrshort{us} respecte l'acronyme \acrfull{invest} \autocite{alliance_what_2015}.
Elles sont également rédigées en tant que "Job Stories" afin de changer la perception des besoins et ainsi remettre au centre ce qui doit être fait pour le client \autocite{klement_replacing_2018}.

\begin{longtable}{|p{0.3\textwidth}|p{0.65\textwidth}|}
\caption{\label{tab:features_overview} Vue structurée des fonctionnalités} \\
\hline
\textbf{Feature} & \textbf{Job Story} \\
\hline
\endfirsthead

\multicolumn{2}{c}{\textit{Suite du tableau \ref{tab:features_overview}}} \\
\hline
\textbf{Feature} & \textbf{Job Story} \\
\hline
\endhead

\hline
\multicolumn{2}{|c|}{\textbf{Epic 1 – Expérience utilisateur et personnalisation de l’interface}} \\
\hline
\multirow{2}{=}{\textbf{1.1 } – Thématisation de l’application} & \textbf{1.1.1}  Lorsque je lance l’application pour la première fois, je veux pouvoir choisir un thème global (clair, sombre, aventure, Lego Mindset, rose) afin de personnaliser l’ambiance visuelle selon mes préférences. \\
 & \textbf{1.1.2}  Lorsque j’utilise l’application, je veux que le thème sélectionné s’applique uniformément à tous les écrans. \\
\hline
\multirow{3}{=}{\textbf{1.2 } – Adaptation à l’âge de l’utilisateur} & \textbf{1.2.1}  Lorsque je configure l’application, je veux pouvoir indiquer mon âge pour adapter l’interface à ma tranche (6–12 ou 13–18 ans). \\
 & \textbf{1.2.2}  Lorsque j’ai moins de 12 ans, je veux une interface simplifiée avec pictogrammes, blocs agrandis et peu de texte. \\
 & \textbf{1.2.3}  Lorsque j’ai 13 ans ou plus, je veux une interface plus avancée avec affichage de la console et du code Python généré. \\
\hline
\multirow{2}{=}{\textbf{1.3 } – Multilinguisme} & \textbf{1.3.1}  Lorsque je configure l’application, je veux pouvoir choisir ma langue (FR, EN, DE, GE) afin de comprendre tous les textes affichés. \\
 & \textbf{1.3.2}  Lorsque je change la langue dans les paramètres, je veux que tous les textes de l’application soient automatiquement mis à jour. \\
\hline

\multicolumn{2}{|c|}{\textbf{Epic 2 – Connexion au robot et configuration matérielle}} \\
\hline
\multirow{3}{=}{\textbf{2.1 } – Sélection et gestion des robots} & \textbf{2.1.1}  Lorsque je souhaite me connecter à un robot, je veux pouvoir le choisir dans une liste de robots disponibles. \\
 & \textbf{2.1.2}  Lorsque je sélectionne un robot, je veux pouvoir établir, tester ou interrompre une connexion. \\
 & \textbf{2.1.3}  Lorsque j’ajoute ou retire un robot, je veux que les informations soient sauvegardées dans un fichier statique \texttt{robots.json}. \\
\hline
\multirow{2}{=}{\textbf{2.2 } – Choix du mode de fonctionnement} & \textbf{2.2.1}  Lorsque j’utilise l’application, je veux pouvoir choisir entre le mode "navigation" ou "exploration" pour définir le comportement général du robot. \\
 & \textbf{2.2.2}  Lorsque mes besoins évoluent, je veux pouvoir changer de mode à tout moment depuis les paramètres. \\
\hline

\multicolumn{2}{|c|}{\textbf{Epic 3 – Programmation visuelle et interaction robot}} \\
\hline
\multirow{4}{=}{\textbf{3.1 } – Construction de scripts visuels} & \textbf{3.1.1}  Lorsque je programme, je veux accéder à des blocs classés par catégories pour créer un script. \\
 & \textbf{3.1.2}  Lorsque je glisse un bloc dans la zone de script, je veux qu’il soit intégré dans l’ordre d’exécution. \\
 & \textbf{3.1.3}  Lorsque j’exécute un script, je veux pouvoir utiliser les boutons Play, Pause et Stop. \\
 & \textbf{3.1.4}  Lorsque le script est en cours d’exécution, je veux que le bloc en cours change de couleur pour savoir ce que le robot fait. \\
\hline
\multirow{3}{=}{\textbf{3.2 } – Console et visualisation du code} & \textbf{3.2.1}  Lorsque j’utilise le mode avancé, je veux voir une console avec les retours du robot en temps réel. \\
 & \textbf{3.2.2}  Lorsque je souhaite, je veux pouvoir afficher le code Python correspondant au script visuel dans une fenêtre dédiée. \\
 & \textbf{3.2.3}  Lorsque je suis en mode simple, je ne veux pas voir ces éléments pour ne pas être submergé. \\
\hline
\textbf{3.3 } – Paramètres dynamiques & \textbf{3.3.1}  Lorsque je navigue dans l’application, je veux accéder à une page de paramètres me permettant de modifier à tout moment le thème, l’âge, la langue, le robot et le mode. \\
\hline

\multicolumn{2}{|c|}{\textbf{Epic 4 – Exécution côté robot et communication réseau}} \\
\hline
\multirow{2}{=}{\textbf{4.1 } – Communication WebSocket bidirectionnelle} & \textbf{4.1.1}  Lorsque je me connecte au robot, je veux établir une communication WebSocket stable avec l’interface. \\
 & \textbf{4.1.2}  Lorsque j’envoie un script, je veux que le robot reçoive chaque instruction et réponde avec des états ou des erreurs. \\
\hline
\multirow{2}{=}{\textbf{4.2 } – Traitement et exécution de scripts} & \textbf{4.2.1}  Lorsque je reçois un script, je veux exécuter chaque commande (déplacement, lecture capteurs, LEDs…). \\
 & \textbf{4.2.2}  Lorsque je détecte un obstacle ou un événement, je veux pouvoir adapter l’exécution automatiquement (si prévu). \\
\hline
\multirow{2}{=}{\textbf{4.3 } – Gestion des états du robot} & \textbf{4.3.1}  Lorsque l’état du robot change (erreur, en pause, en exécution), je veux déclencher des LEDs et sons associés. \\
 & \textbf{4.3.2}  Lorsque l’état change, je veux envoyer un feedback structuré à l’interface pour l’informer. \\
\hline

\multicolumn{2}{|c|}{\textbf{Epic 5 – Scénarii multi-robots}\footnote{L'Epic 5 représente des perspectives de développement ultérieur, en dehors du périmètre de ce projet.
Il vise à élargir les possibilités pédagogiques par l'introduction de comportements collaboratifs ou compétitifs entre robots.}} \\
\hline
\multirow{2}{=}{\textbf{5.1 } – Coordination entre plusieurs robots} & \textbf{5.1.1}  Lorsque plusieurs robots sont utilisés en simultané, je veux qu’ils puissent synchroniser leurs actions pour réaliser des comportements collectifs (formation, évitement, etc.). \\
 & \textbf{5.1.2}  Lorsque les robots partagent un environnement, je veux pouvoir leur assigner des rôles différents (ex. : suiveur, leader) afin d’illustrer des dynamiques de groupe. \\
\hline
\multirow{2}{=}{\textbf{5.2 } – Simulation de comportements interactifs} & \textbf{5.2.1}  Lorsque deux robots sont configurés pour interagir, je veux pouvoir simuler un jeu du type "chasseur et proie" afin d’illustrer des principes de stratégie, détection, et réponse adaptative. \\
 & \textbf{5.2.2}  Lorsque plusieurs robots se croisent, je veux pouvoir observer et ajuster leurs comportements d’évitement et de priorité de passage. \\
\hline

\end{longtable}

