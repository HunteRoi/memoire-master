\subsubsection{Choix technologiques pour le processus embarqué} \label{sec:choix_robot}

En s’appuyant sur le rapport "State of Developer Ecosystem 2024" publié par JetBrains (23262 réponses recueillies sur l’ensemble de l’année) \autocite{noauthor_software_nodate}, et en croisant ces données avec les informations issues de dépôts de code publics, de sites officiels des technologies concernées, etc., les comparaisons suivantes ont pu être établies.

\begin{longtable}{|p{0.25\textwidth}|p{0.325\textwidth}|p{0.325\textwidth}|}
\caption{\label{tab:comparison_robot_languages} Comparaison détaillée des langages de programmation embarqués} \\

\hline
\textbf{Critères \textbackslash\ Langages} & \textbf{Python} & \textbf{C++} \\
\hline
\endfirsthead

\multicolumn{3}{c}{\textit{Suite du tableau \ref{tab:comparison_robot_languages}}} \\
\hline
\textbf{Critères \textbackslash\ Langages} & \textbf{Python} & \textbf{C++} \\
\hline
\endhead

\textbf{Installation sur le robot (/5)} & 
5, interpréteur préinstallé, prêt à l’emploi & 
5, compilateur \texttt{gcc} déjà disponible \\
\hline

\textbf{Facilité d’écriture (/5)} & 
5, syntaxe simple et intuitive, adaptée aux enfants & 
3, syntaxe plus rigide et verbosité plus élevée \\
\hline

\textbf{Performance (/5)} & 
3, suffisant pour l’éducation mais pas optimal en temps réel & 
5, hautes performances, proche du matériel \\
\hline

\textbf{Consommation mémoire (/5)} & 
3, interpréteur gourmand sur petit matériel & 
5, faible empreinte mémoire une fois compilé \\
\hline

\textbf{Support robotique (/5)} & 
5, très utilisé en robotique éducative (OpenCV, PySerial...) & 
5, nombreuses bibliothèques performantes (ROS, etc.) \\
\hline

\textbf{Courbe d’apprentissage (/5)} & 
5, idéal pour les débutants et rapide à maîtriser & 
3, courbe abrupte, concepts avancés requis \\
\hline

\textbf{Modernité (/5)} & 
5, tendance actuelle, très utilisé en IA et éducation & 
4, mature et puissant, mais syntaxe parfois datée \\
\hline

\textbf{Total (/35)} & 
\textbf{31} & 
\textbf{30} \\
\hline

\textbf{Score (\%)} & 
89\% & 
86\% \\
\hline

\end{longtable}


\paragraph{Critères d’évaluation}
Les langages sont comparés selon sept critères clés, chacun noté sur 5 points. 
Voici les définitions :

\begin{longtable}{|p{0.30\textwidth}|p{0.30\textwidth}|p{0.30\textwidth}|}
\caption{\label{tab:criteria_robot_languages} Grille d’évaluation des langages embarqués} \\

\hline
\textbf{Critère} & \textbf{Définition} & \textbf{Échelle (1 à 5)} \\
\hline
\endfirsthead

\multicolumn{3}{c}{\textit{Suite du tableau \ref{tab:criteria_robot_languages}}} \\
\hline
\textbf{Critère} & \textbf{Définition} & \textbf{Échelle (1 à 5)} \\
\hline
\endhead

Déjà installé & Présence native de l’interpréteur ou compilateur sur le robot. & 
\begin{itemize}
  \item 5 : oui sans installation
  \item 1 : non disponible
\end{itemize} \\
\hline

Facilité d’écriture & Simplicité syntaxique et expérience de développement. & 
\begin{itemize}
  \item 5 : très facile
  \item 1 : complexe
\end{itemize} \\
\hline

Performance & Rapidité d’exécution et efficacité à faible charge. & 
\begin{itemize}
  \item 5 : très performant
  \item 1 : lent
\end{itemize} \\
\hline

Consommation mémoire & Empreinte mémoire lors de l’usage sur le robot. & 
\begin{itemize}
  \item 5 : très faible
  \item 1 : très élevée
\end{itemize} \\
\hline

Support robotique & Disponibilité de bibliothèques, documentation spécialisée. & 
\begin{itemize}
  \item 5 : excellent support
  \item 1 : support limité
\end{itemize} \\
\hline

Courbe d’apprentissage & Rapidité d’adoption par des utilisateurs non experts. & 
\begin{itemize}
  \item 5 : très douce
  \item 1 : très abrupte
\end{itemize} \\
\hline

Modernité & Popularité récente et adoption dans les projets actuels. & 
\begin{itemize}
  \item 5 : très moderne
  \item 1 : obsolète
\end{itemize} \\
\hline

\end{longtable}
