\subsubsection{Choix technologiques concernant le processus embarqué dans le robot} \label{sec:choix_robot}
Deux langages sont disponibles nativement dans l'e-puck2 :
\begin{itemize}
    \item \textbf{Python} – déjà installé, facile à manipuler, très utilisé en robotique éducative.
    \item \textbf{C++} – compilable avec \textit{gcc} déjà disponible sur les robots, performant mais plus complexe.
\end{itemize}

\begin{longtable}[H]{|p{0.22\textwidth}|p{0.26\textwidth}|p{0.26\textwidth}|}
\caption{\label{tab:comparison_robot_languages} Comparaison des langages utilisables sur le robot} \\

\hline
\textbf{Critères} & \textbf{Python} & \textbf{C++} \\
\endfirsthead

\hline
\textbf{Déjà installé sur le robot} & Oui & Oui (compilateur) \\
\hline
\textbf{Facilité d’écriture} & Très élevée & Moyenne à faible \\
\hline
\textbf{Performance} & Moyenne & Élevée \\
\hline
\textbf{Consommation mémoire} & Moyenne & Faible \\
\hline
\textbf{Support robotique} & Excellent (OpenCV, PySerial, etc.) & Excellent (ROS, libs natives) \\
\hline
\textbf{Courbe d’apprentissage} & Douce & Abrupte \\
\hline
\end{longtable}
