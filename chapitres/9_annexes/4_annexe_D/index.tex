\section{Annexe - Références et compléments} \label{sec:annexes_d}
Comme expliqué dans la bibliographie, ce document ainsi que son prototype applicatif ont puisé leurs reformulations, corrections orthographiques, aides typographiques, etc. dans les \acrshort{ia} génératrices telles que ChatGPT, Copilot, Claude AI, ...

Cette annexe vise à expliciter de manière entièrement transparente l'utilisation de cette technologie dans le cadre de la rédaction de ce document et du développement du prototype.

\subsection{Rédaction du travail}
\subsubsection{Informations complémentaires}
L'\acrfull{ia} employée majoritairement dans ce travail est ChatGPT dans ses versions o3, GPT-4o, o4-mini et GPT-5.

La méthodologie employée est de passer par un prompt qui donne un contexte clair et structuré et limite les dérives de cette intelligence artificielle générative tout en exploitant ses capacités.

Ainsi, la reformulation du texte, les manières de faire convoyer les idées et aussi la possibilité de challenger mes décisions sont passés dans cette moulinette numérique.

Cependant, il est important de mentionner que tout le premier jet de ce travail a bel et bien été rédigé par moi-même (parfois de manière un peu confuse) et que toute recherche de références n'a jamais été réalisée par l'\acrshort{ia} sans être corroborée par des sources externes provenant de mes propres recherches.

\subsubsection{Prompt employé}
Mon premier prompt à l'intention de l'\acrshort{ia} a été formulé de manière à donner du contexte sur la tâche à réaliser et à limiter ce que l'\acrshort{ia} peut réaliser dans ce domaine.

Voici un copier-coller du prompt original: 
\begin{boxitup}
Je suis un étudiant de 25 ans qui écrit un mémoire pour son master en architecture des systèmes informatiques. J'ai également de l'expérience en informatique car j'ai travaillé 4 ans dans le domaine en tant que développeur.

Voici l'ensemble des consignes du guide pour la rédaction du travail de fin d'étude à écrire:

\begin{itemize}
    \item structure du TFE: le travail doit être composé d'une page de garde, d'un avant-propos, d'un sommaire, d'une liste de symboles (si nécessaire), d'une introduction, d'un état de l'art, d'une conclusion, d'une liste de sources (bibliographie, sitographie, etc.) et d'annexes éventuelles ;

    \item volume attendu: au minimum 50 pages hors annexes et bibliographie ;

    \item langue: le travail rédigé en français ;

    \item règles typographiques: la police est lisible, en interligne simple, le texte justifié, les pages numérotées, et les sections clairement définies ;

    \item citations et sources: références au format IEEE ;

    \item annexes: ne contiennent que des documents utiles à la compréhension du travail (du code, des extraits d'analyse, des captures d'écran, des questionnaires, etc.).
\end{itemize}

J'écris ce mémoire en utilisant Overleaf, un site de compilation en ligne pour LaTeX.
Ainsi, priorise tes réponses dans le format LaTeX si je te demande de reformuler.

Lorsque j'interagirai avec toi, tu n'auras le droit que d'agir en tant que:
\begin{enumerate}
    \item assistant linguistique (amélioration de textes, correction orthographique et grammaticale, relecture et réécriture pour améliorer la lecture et compréhension) ;

    \item assistant à la recherche d'informations (en sourcant toujours tes informations, et en priorisant les informations qui proviennent de sources sûres comme ResearchGate ou GoogleScholar) ;

    \item assistant à la créativité et la génération d'idées (e.g. brainstorming, organisation d'informations, conception d'ébauche, etc.).
\end{enumerate}
\end{boxitup}

Tous les prompts suivants ont été formulés en continuité de celui-ci, afin de tirer parti de l'\acrshort{ia} au mieux.
L'encadré suivant donne un exemple de prompt que j'ai envoyé à ChatGPT pour plusieurs chapitres de ce travail.
\begin{boxitup}
Tu peux commencer par m'aider à reformuler le paragraphe suivant: $[...]$.
\end{boxitup}

\subsubsection{Réponses attendues et points d'attention}
Il est toujours intéressant de voir comment l'\acrshort{ia} répond aux différentes demandes effectuées.
Il est tout de même très important de vérifier ses dires (pour éviter de dénaturer le texte et les intentions que j'y ai glissées ou perdre le fil rouge de celui-ci), ses sources et informations quand elle juge adéquat d'en rajouter (toujours à la demande), etc.

Aucun contenu de ce travail n'a été copié-collé sans être adapté, que cela soit un simple mot pour que la force de la phrase soit différente, ou la ponctuation, voire même parfois des répétitions de mots ou des formulations de phrases dont le ton ne s'accordait pas avec le reste du travail ou ma manière d'écrire.

J'ai particulièrement été attentif à la manière dont l'\acrshort{ia} reformulait mes phrases et à tout ce qu'elle ajoutait à mes écrits quand je le lui demandais.

\subsection{Développement du prototype applicatif}
\subsubsection{Informations complémentaires}
Les \acrshort{ia}s employées majoritairement dans ce travail sont Copilot dans sa version gratuite (qui emploie notamment ChatGPT) et Claude AI.

La méthodologie employée est de permettre à l'outil de réaliser de l'autocomplétion selon le contexte, et de passer par un prompt pour réaliser des tâches fatiguantes (e.g. traduire un fichier complet, réaliser un \textit{refactoring} en divisant une grosse méthode publique en plus petites méthodes privées, etc.).

Ainsi, le code employé est un mélange entre des inspirations personnelles et des \textit{snippets} de code de l'\acrshort{ia}.

Cependant, il est important de mentionner qu'à part les fichiers de traduction ainsi que le composant qui décrit un robot e-puck2 en SVG, le code a bel et bien été rédigé par moi-même, boosté par ces outils.

\subsubsection{Prompt employé}
Les prompts employés pour Copilot et Claude AI sont assez contextualisés.
Le plus concret est celui qui m'a permis de traduire les fichiers JSON en plusieurs langues:
\begin{boxitup}
Look at the JSON structure of the /locales/en.json file.
Once you understand its structure, create other translation files for French (fr.json), Dutch (nl.json), German (de.json) and Italian (it.json) in the same folder.
\end{boxitup}

\subsubsection{Réponses attendues et points d'attention}
Dans le cadre d'un développement informatique, il est d'autant plus important de vérifier que l'\acrshort{ia} ne prend pas des initiatives ou décisions insensées, car développer une usine à gaz pour faire une simple soustraction est inutile.

L'esprit critique et la réflexion sont de mise dans un contexte tel que celui-ci.
Il faut pouvoir remettre en question ses propres décisions mais également vérifier que l'outil employé ne déborde pas dans ses modifications du code.

J'ai particulièrement été attentif à la manière dont l'\acrshort{ia} utilisait les méthodes disponibles pour éviter la duplication.
J'ai aussi été très pointilleux sur les noms de variable, fonctions et classes qu'elle pouvait générer ainsi que la taille des fichiers pour éviter de submerger le développeur \autocites{mark_seemann_fractal_2021}{mark_seemann_code_2021}.
