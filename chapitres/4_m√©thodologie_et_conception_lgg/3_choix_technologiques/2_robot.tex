\subsubsection{Choix du langage pour le code embarqué sur le robot} \label{sec:robot_code}

Le robot e-puck2 utilisé dans ce projet est équipé d’un Raspberry Pi Zero W tournant sous Linux.
Le \autoref{tab:comparison_robot_languages} compare les choix possibles dans la \autoref{sec:choix_robot}.
Le choix s’oriente naturellement vers \textbf{Python}, car il est :
\begin{itemize}
    \item déjà installé et supporté,
    \item facile à lire et à écrire pour des étudiants,
    \item compatible avec de nombreux modules utiles (exécution dynamique, socket, etc.),
    \item suffisamment performant dans le cadre d’un robot éducatif.
\end{itemize}

C++ reste une option à envisager pour les composants critiques (communication en temps réel, pilotage précis), mais Python constitue la base idéale pour exécuter les scripts utilisateurs envoyés depuis l’interface cliente.
