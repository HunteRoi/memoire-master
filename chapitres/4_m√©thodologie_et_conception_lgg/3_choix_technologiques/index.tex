\subsection{Choix technologiques} \label{sec:choix_technos}
Dans le contexte de ce mémoire, l'application vise à offrir une interface de programmation intuitive pour contrôler un ou plusieurs robots e-puck2 via le réseau, avec la possibilité de gérer du code, des fichiers, et des interactions en temps réel.

Deux volets technologiques sont à considérer :  
\begin{enumerate}
    \item \textbf{Le client lourd (interface utilisateur)}, qui sera utilisé sur des ordinateurs personnels.
    \item \textbf{Le code embarqué sur le robot}, qui exécutera les instructions générées à distance.
\end{enumerate}

Les décisions techniques doivent donc équilibrer performance, maintenabilité, simplicité de déploiement, et compatibilité avec les contraintes matérielles du robot.

\subimport{}{1_client}
\subimport{}{2_robot}
\subimport{}{3_conclusion}
