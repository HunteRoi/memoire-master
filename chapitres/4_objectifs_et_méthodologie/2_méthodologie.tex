\subsection{Méthodologie} \label{sec:méthodologie}

Ce projet combine des dimensions techniques, pédagogiques et exploratoires et adopte donc une approche Agile, et plus précisément la méthode \textbf{Scrum}.  
Les fonctionnalités visées sont présentées dans la \autoref{sec:annexes_b}.

L'organisation temporelle du projet est illustrée par un diagramme de Gantt (\autoref{fig:complete_gantt_chart}) complet également disponible en \autoref{sec:annexes_b}, qui couvre l’ensemble du processus de conception et de rédaction.
La \autoref{fig:gantt_prototype_phases_detailed} présente sous forme détaillée les phases de création du prototype technique.

\begin{figure}[H]
    \centering
    \scalebox{0.5}{
        \begin{ganttchart}[
            hgrid,
            vgrid,
            inline,
            time slot format=isodate,
            bar height=0.5,
            bar/.style={fill=blue!30},
            bar label font=\small\bfseries,
            group right shift=0,
            group top shift=0.7,
            group height=0.3,
            group peaks height=0.2,
            group label font=\bfseries\small,
        ]{2025-06-01}{2025-08-03}
            \gantttitlecalendar{month=name, day} \\

            % Groupe 1 : Conception
            \ganttgroup{Conception}{2025-06-01}{2025-07-20} \\
            \ganttbar{Analyse des besoins techniques}{2025-06-01}{2025-06-10} \\
            \ganttbar{Spécification des fonctionnalités}{2025-06-11}{2025-06-17} \\
            \ganttbar{Modélisation de l’architecture logicielle}{2025-06-18}{2025-06-30} \\
            \ganttbar{Conception de l’interface utilisateur}{2025-07-01}{2025-07-17} \\
            \ganttbar{Préparation des tests unitaires}{2025-07-18}{2025-07-20} \\

            % Groupe 2 : Développement - Firmware
            \ganttgroup{Développement firmware (robot)}{2025-07-21}{2025-07-27} \\
            \ganttbar{Script Python principal (boucle)}{2025-07-21}{2025-07-22} \\
            \ganttbar{Connexion WebSocket avec PC}{2025-07-22}{2025-07-23} \\
            \ganttbar{Exécution des actions}{2025-07-23}{2025-07-24} \\
            \ganttbar{Déplacement aléatoire}{2025-07-24}{2025-07-25} \\
            \ganttbar{Système de codes d’erreur}{2025-07-25}{2025-07-26} \\
            \ganttbar{LEDs + sons selon état}{2025-07-26}{2025-07-27} \\

            % Groupe 3 : Développement - Interface
            \ganttgroup{Développement interface (Electron)}{2025-07-21}{2025-07-31} \\
            \ganttbar{Page de choix (mode)}{2025-07-21}{2025-07-22} \\
            \ganttbar{Layout blocs + console}{2025-07-22}{2025-07-24} \\
            \ganttbar{Catégories de blocs}{2025-07-24}{2025-07-25} \\
            \ganttbar{Console + boutons play/stop}{2025-07-25}{2025-07-26} \\
            \ganttbar{Écran paramètres (modal)}{2025-07-26}{2025-07-27} \\
            \ganttbar{Formulaire TCP/IP + test}{2025-07-27}{2025-07-28} \\
            \ganttbar{Traduction blocs → Python}{2025-07-28}{2025-07-31} \\

            % Groupe 4 : Intégration & Tests
            \ganttgroup{Intégration et tests}{2025-07-25}{2025-08-03} \\
            \ganttbar{Connexion client-robot}{2025-07-25}{2025-08-01} \\
            \ganttbar{Tests d’intégration globaux}{2025-08-01}{2025-08-03} \\
        \end{ganttchart}
    }
    \caption{\label{fig:gantt_prototype_phases_detailed} Planning détaillé du développement du prototype}
\end{figure}
