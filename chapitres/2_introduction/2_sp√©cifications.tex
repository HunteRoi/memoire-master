\subsection{Spécifications} \label{sec:specs}

L’objectif d'une telle interface est de permettre la réalisation de trois grandes catégories d’actions, définies comme suit par le Dr. Tuci :  

\begin{itemize}
    \item \textbf{Navigation} : Exécution de parcours prédéfinis, comme la traversée d’un labyrinthe.
    \item \textbf{Exploration} : Recherche et identification d’éléments spécifiques, tels que des dalles de couleur.
    \item \textbf{Observation de comportements collectifs} : Analyse de mouvements co-déterminés, par exemple dans un scénario de type "jeu du chat et de la souris".
\end{itemize}

Afin de répondre à ces exigences, les éléments suivants constituent les spécifications fonctionnelles de l’interface à développer :

\begin{enumerate}
    \item Assurer une communication instantanée avec le robot e-puck pour permettre un développement en temps réel.
    
    \item Permettre la configuration et la modification de la couleur des LED du robot afin de faciliter son identification visuelle.

    \item Permettre une configuration flexible des déplacements du robot :  
        \begin{itemize}
            \item Mouvements aléatoires.  
            \item Mouvements pré-programmés, exécutés en une seule fois ou ajustés en temps réel, en fonction de la nouvelle position du robot.  
        \end{itemize}  

    \item Définir une vitesse de déplacement (réglable sur trois niveaux) et permettre un paramétrage de la distance parcourue.  

    \item Intégrer une option d’évitement d’obstacles :  
        \begin{itemize}
            \item En mode aléatoire, le robot doit ajuster sa trajectoire pour contourner l’obstacle.  
            \item En mode pré-programmé, le robot doit stopper son mouvement face à un obstacle et émettre un message d'erreur.  
        \end{itemize}  

    \item Implémenter un système de reconnaissance des zones de couleur, où le robot évolue sur une arène noire et doit identifier des cases blanches.

    \item Intégrer un système de retour visuel et sonore indiquant l’état du robot (e.g. l'affichage en rouge lorsqu’il est bloqué, en vert lorsqu’un objectif est atteint, etc.). 
\end{enumerate}  
