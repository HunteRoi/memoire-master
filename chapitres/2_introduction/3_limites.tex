\subsection{Limites} \label{sec:limites}

L’objectif de ce mémoire étant relativement vaste, il est nécessaire d’en définir les limites afin de garantir des résultats concrets et exploitables.  

Dans cette première version, l’interface proposera un ensemble restreint d’instructions.
L’extension du jeu d’instructions et l’amélioration de l’interface, notamment par la possibilité de modéliser un environnement spécifique dans lequel évolue le robot (délimitations physiques, obstacles, etc.), pourront faire l’objet de développements futurs.

Quant aux spécifications données au chapitre \ref{sec:specs}, l'objectif premier visé est de permettre un contrôle minimal d'un seul robot dans une interface à la fois simple d’utilisation et évolutive, afin de garantir la maintenabilité de l'interface et d’éviter que les futures évolutions de celle-ci ne deviennent une contrainte majeure pour les développeurs amenés à travailler dessus.

Par ailleurs, la compatibilité matérielle constitue un enjeu central.
Ce travail s’efforcera donc d’adopter une approche générique en matière d’interface logicielle afin de faciliter d’éventuelles évolutions, tout en ciblant exclusivement les robots e-puck.

Enfin, il est primordial de prendre en compte les spécificités du public cible, constitué d’élèves du primaire et du secondaire âgés de 6 à 18 ans, ainsi que les contraintes sociotechniques propres à un projet de cette envergure, telles que l’intuitivité de l’interface, l’accessibilité des concepts et l’interactivité de l’apprentissage.  
