\subsection{Objectifs définis} \label{sec:objectifs_finaux}

Ce chapitre a pour vocation de synthétiser les conclusions et orientations issues des sections précédentes. 
Il constitue un point de passage important avant d’aborder les aspects concrets du développement.

En réponse aux besoins exprimés dans les chapitres \ref{sec:limites}, \ref{sec:specs} et \ref{sec:non-functional_specs}, et en cohérence avec les conclusions du chapitre \ref{sec:conclusion_recherches}, l’objectif principal de cette version est la \textbf{conception d’une application ludique}, inspirée des codes du jeu vidéo, afin de favoriser l’apprentissage par l’expérimentation et de stimuler la motivation des élèves.

L’interface proposée permettra la \textbf{connexion à un seul robot} dans un premier temps, bien que les choix techniques soient guidés par la volonté d’assurer une extensibilité future à un ensemble de robots.

Cette application sera développée de manière à être \textbf{employée sur un ordinateur} (malgré le potentiel intérêt pour les appareils mobiles comme expliqué dans la \autoref{sec:interfaces_existantes}).

Elle embarquera un \textbf{système de programmation par blocs}, qui \textbf{traduira} du \textbf{langage visuel} vers le \textbf{Python}.
Chaque bloc ajouté au script sera \textbf{exécuté de manière interactive} sur le robot, dans l’ordre de leur empilement, favorisant ainsi une compréhension progressive et visuelle du déroulement du script.

Deux modes d’utilisation sont ciblés: la \textbf{navigation} et l’\textbf{exploration}. 
Pour répondre à ces exigences, le robot e-puck2 devra:
\begin{itemize}
    \item \textbf{se déplacer} dans toutes les directions de manière contrôlée ou aléatoire ;
    \item \textbf{produire des signaux} visuels (\acrshort{led}s) et sonores pour refléter son état ;
    \item \textbf{détecter les couleurs} du sol via ses capteurs.
\end{itemize}

\begin{figure}[H]
    \centering
    \scalebox{0.7}{
    \begin{tikzpicture}
        \begin{umlsystem}[x=0]{PuckLab}           
            \umlusecase[name=blocs,y=-2]{Programmer des blocs}
                \umlusecase[name=python,y=-2,x=10]{Traduire les blocs en script}
            
            \umlusecase[name=control_script,y=-4]{Gérer l'exécution du script}
                \umlusecase[name=move,y=-4,x=10]{Contrôler les mouvements}
                \umlusecase[name=colours,y=-6,x=10]{Contrôler les capteurs de couleur}
                \umlusecase[name=sounds,y=-8,x=10]{Contrôler les haut-parleurs}
            
            \umlusecase[name=connect,y=-10]{Se connecter au robot}
                \umlusecase[name=start_connection,y=-10,x=10]{Ouvrir une connexion}
                \umlusecase[name=led,y=-12,x=10]{Contrôler les \acrshort{led}s}

            \umlusecase[name=disconnect,y=-14]{Se déconnecter du robot}
                \umlusecase[name=stop_connection,y=-14,x=10]{Fermer une connexion}
        \end{umlsystem}

        \umlactor[x=-6,y=-8]{Utilisateur}

        \umlassoc{Utilisateur}{blocs}
        \umlassoc{Utilisateur}{control_script}
        \umlassoc{Utilisateur}{connect}
        \umlassoc{Utilisateur}{disconnect}

        \umlinclude{blocs}{python}
        \umlinclude{control_script}{move}
        \umlinclude{control_script}{colours}
        \umlinclude{control_script}{sounds}
        \umlinclude{control_script}{led}
        
        \umlinclude{connect}{start_connection}
        \umlinclude{connect}{led}
        \umlinclude{disconnect}{stop_connection}
        \umlinclude{disconnect}{led}
        
    \end{tikzpicture}
    }
    \caption{Diagramme de cas d'utilisation de l'application PuckLab}
    \label{fig:usecase-pucklab}
\end{figure}
