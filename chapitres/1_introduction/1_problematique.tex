\subsection{Problématique} \label{sec:problématiques}

Avec l’intérêt croissant des jeunes pour les domaines informatiques tels que la robotique, l’intelligence artificielle ou encore la création de jeux vidéo, une initiation à ces sujets constitue une opportunité pour l’Université de Namur de promouvoir des formations pour ces métiers en pleine évolution et de susciter la curiosité des nouvelles générations.  

Toutefois, lors de chaque événement (e.g. le SIEP, le Printemps des Sciences, les Journées Portes Ouvertes), l’UNamur est confrontée à une difficulté récurrente : proposer des démonstrations attrayantes et motivantes pour les potentiels futurs étudiants, c'est-à-dire les actuels élèves du primaire et du secondaire. 
En effet, bien que les robots e-puck\footnote{Un e-puck est un petit robot rond doté de deux roues, d'une carrosserie en plastique, de plusieurs capteurs infra-rouges et d'une caméra. Une description de ces machines est détaillée en \autoref{sec:robot_epuck}.} permettent d’illustrer concrètement des concepts liés à la programmation, leur utilisation requiert des compétences techniques qui font défaut à ces non-initiés.

Ce mémoire vise donc à concevoir une interface intuitive permettant au public cible d’interagir avec ces robots sans nécessiter de connaissances préalables en programmation ou en robotique.  
