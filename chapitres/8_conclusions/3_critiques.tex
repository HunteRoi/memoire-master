\subsection{Critiques du travail réalisé} \label{sec:critiques}

L’intégration avec le robot physique ainsi que les tests d’interopérabilité entre l’interface et le serveur robot n’ont pas encore pu être menés à bien.
Ce contretemps a également empêché la réalisation de la phase de tests utilisateurs initialement prévue, laquelle visait à recueillir des retours constructifs et à identifier des pistes d’amélioration avant la finalisation du projet\footnote{Cette limitation justifie le déplacement de ce volet vers la \autoref{sec:annexes_a}.}.

Le travail accompli présente néanmoins une réelle profondeur, avec une volonté affirmée de concevoir une application sérieuse, accessible et extensible.  
L’analyse architecturale, en particulier, a été menée avec soin et orientée vers l’évolutivité, renforcée par le choix de rendre le projet open source afin de favoriser les contributions futures.

Cependant, certains points auraient pu être approfondis :
\begin{itemize}
    \item \textbf{Langage visuel et génération Python} :  
    La partie centrale du projet — le langage visuel et sa traduction en Python — aurait gagné à être documentée plus précisément.  
    Une liste exhaustive des blocs disponibles, accompagnée de leur traduction Python correspondante, aurait facilité à la fois le développement, la maintenance et la compréhension de cette couche.
    
    \item \textbf{Validation technique} :  
    L’absence de tests intégrés et d’essais sur robot réel limite la validation pratique des choix techniques et des performances réelles.
    
    \item \textbf{Expérience utilisateur} :  
    Faute de tests utilisateurs, certaines hypothèses d’UX/UI restent théoriques. Une étude ergonomique appuyée par des retours concrets aurait permis d’orienter plus précisément l’interface.
    
    \item \textbf{Couverture fonctionnelle} :  
    Certaines fonctionnalités prévues (gestion avancée des scripts, synchronisation inter-robots, profils persistants) restent à l’état de perspectives, et leur absence limite légèrement la démonstration complète du potentiel de l’outil.
    
    \item \textbf{Qualité documentaire} :  
    Bien que le mémoire présente un contenu riche, certaines sections gagneraient à être structurées de manière plus concise pour améliorer la lisibilité.  
    Des schémas supplémentaires (par exemple pour illustrer le flux complet de traduction blocs → Python → exécution robot) auraient pu clarifier encore davantage les concepts.
\end{itemize}

Le travail effectué pose de bonnes bases techniques et démontre une maîtrise claire des principes d’architecture et des bonnes pratiques de développement.
Cependant, la validation par l’usage réel et la couverture fonctionnelle par des tests restent les principaux axes à travailler pour atteindre une maturité pleinement opérationnelle.
