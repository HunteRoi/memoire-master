\subsection{Synthèse des contributions} \label{sec:synthese_contribs}

Au terme de ce travail, l'application est pleinement exécutable sur toute machine compatible.

\paragraph{Côté interface}  
L’ensemble des fonctionnalités prévues a été implémenté et est opérationnel :
\begin{itemize}
    \item page de sélection du thème, appliquée à l’ensemble de l’interface ;
    \item page de sélection de l’âge, influençant l’affichage de l’éditeur visuel ;
    \item page de sélection du robot, avec gestion de la connexion ;
    \item page de sélection du mode, permettant le filtrage et le tri des blocs visuels ;
    \item éditeur de programmation visuelle en glisser-déposer, accompagné d’une console et d’un visualisateur du script Python généré ;
    \item page de paramètres pour modifier à tout moment les options choisies ;
    \item support multilingue complet ;
    \item ergonomie soignée, avec notamment la navigation clavier et l’usage d’icônes/pictogrammes cohérents.
\end{itemize}

\paragraph{Côté robot}  
La structure logicielle et les principales implémentations nécessaires au fonctionnement du serveur embarqué ont été réalisées, incluant :
\begin{itemize}
    \item la communication par WebSocket ;
    \item les liaisons matérielles via les bibliothèques Python spécifiques à l’e-puck.
\end{itemize}
Ces éléments constituent le socle fonctionnel principal du développement Python.
