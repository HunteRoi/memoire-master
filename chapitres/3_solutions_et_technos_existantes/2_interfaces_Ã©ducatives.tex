\subsection{Interfaces éducatives} \label{sec:interfaces_éducatives}

Les outils éducatifs existants répondent à des besoins pédagogiques variés.
Le tableau \ref{tab:comparison_tools} (en \autoref{sec:annexes_a}) résume leurs principales caractéristiques.

Il existe plusieurs outils et environnements logiciels qui illustrent la diversité des approches pédagogiques en robotique et programmation :

\begin{itemize}
    \item \textbf{Simulation robotique avancée} : des plateformes comme \textit{CoppeliaSim (V-REP)} et \textit{Webots} permettent de modéliser des scénarios robotiques complexes dans des environnements réalistes.
    
    \item \textbf{Programmation visuelle} : des outils tels que \textit{Blockly}, \textit{PyBricks}, \textit{ThymioSuite} et \textit{Unreal Engine Blueprints} offrent une approche intuitive de la logique algorithmique, facilitant l’apprentissage pour les débutants grâce à des interfaces à base de blocs ou de nœuds.

    \item \textbf{Programmation textuelle} : des environnements comme \textit{Arduino IDE} et \textit{Open Roberta Lab} (en mode texte) permettent une écriture directe du code, souvent à destination d’un public intermédiaire à avancé.

    \item \textbf{Modélisation et conception électronique} : des outils comme \textit{Tinkercad} et \textit{Processing} sont utilisés pour représenter et tester des circuits électroniques ou des concepts visuels de manière interactive (e.g. par glisser-déposer).
\end{itemize}

L'écosystème des outils éducatifs en programmation et robotique est riche et diversifié, chaque technologie présentant des avantages en fonction du public cible et des objectifs pédagogiques.
Associer plusieurs de ces outils permet de proposer des parcours d’apprentissage progressifs, depuis les bases jusqu’aux projets complexes de robotique et simulation.

Inspirée de la \textit{ThymioSuite}, l’intégration d’un système capable de convertir une programmation visuelle en un langage de haut niveau comme Aseba apparaît comme une stratégie pertinente.
Cependant, le langage Aseba a été initialement conçu pour exploiter les fonctionnalités de base du robot, et ne prend pas en charge certaines extensions avancées de l’e-puck2. 
Cela inclut notamment des dispositifs comme la caméra à 360° ou le module de contrôle \textit{Pi-puck}, qui nécessitent un langage plus souple et plus puissant, tel que le C++ ou le Python.

Afin d’assurer une accessibilité optimale et une montée en compétences progressive, une approche hybride semble plus adaptée : 
\begin{enumerate}
    \item \textbf{Sur le plan de l’interface} : combiner la programmation par blocs avec possibilité d'une interface textuelle permettrait d’assurer une transition progressive entre la découverte de la logique algorithmique et la maîtrise de concepts plus avancés.
    
    \item \textbf{Sur le plan technique} : l’intégration de langages comme Python ou C++ au sein du système permettrait une meilleure prise en charge des capteurs et modules complémentaires du robot, offrant ainsi une plus grande flexibilité d’usage.
\end{enumerate}

Cette approche garantirait un équilibre entre simplicité d’utilisation et flexibilité, facilitant l’accès au robot tant dans un cadre éducatif que pour des applications plus avancées.
