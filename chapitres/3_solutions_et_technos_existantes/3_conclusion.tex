\subsection{Conclusion générale} \label{sec:conclusion_recherches}

L’analyse des outils existants pour l’e-puck2 et de ses fonctionnalités a mis en évidence plusieurs constats majeurs :  

\begin{itemize}
    \item Malgré ses capacités avancées, l’e-puck2 reste difficile à prendre en main, notamment pour les débutants, ce qui limite son adoption dans un cadre éducatif.

    \item Les interfaces actuelles manquent de flexibilité et ne prennent pas en charge les extensions récentes du robot, soulignant le besoin d’une solution plus évolutive et modulaire. 

    \item Une approche hybride combinant programmation par blocs et textuelle est essentielle pour garantir une accessibilité optimale et une progression adaptée aux différents niveaux d’apprentissage.

    \item L’utilisation d’un langage visuel traduit vers du C++ ou du Python faciliterait la prise de contrôle du robot et de ses extensions.

    \item L’absence d’applications mobiles dédiées représente une opportunité pour améliorer l’accessibilité et l’ergonomie de l’e-puck2 auprès d’un plus large public.
\end{itemize}
