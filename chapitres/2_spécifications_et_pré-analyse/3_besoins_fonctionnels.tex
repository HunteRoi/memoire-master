\subsection{Besoins fonctionnels} \label{sec:specs}

Afin de répondre à ces exigences, les éléments suivants constituent les spécifications fonctionnelles de l’interface à développer.

\subsubsection{Communication en temps réel} 
Le robot doit pouvoir échanger des données en continu avec le poste de contrôle : réception des commandes et envoi des retours d’état.  
La notion de temps réel est ici définie comme la capacité à réagir dans un délai suffisamment court pour que l’interaction utilisateur-robot reste fluide et fiable.

Cela signifie évidemment que celui-ci doit être connecté à un réseau afin de communiquer avec le poste de contrôle.

\subsubsection{Identification visuelle}
Chaque robot doit pouvoir être identifié facilement, notamment par le biais de la configuration personnalisable de ses \acrshort{led}s (couleurs distinctes entre les robots).

\subsubsection{Configuration flexible des déplacements}
L’utilisateur doit pouvoir ajuster les déplacements selon plusieurs paramètres :
\begin{itemize}
    \item Trois niveaux de vitesse ;
    \item Une définition de la distance à parcourir ;
    \item Deux modes de déplacement : aléatoire ou pré-programmé.
\end{itemize}

\subsubsection{Gestion des obstacles}
Le comportement face aux obstacles doit être adapté au mode de déplacement :
\begin{itemize}
    \item En mode aléatoire : le robot ajuste sa trajectoire pour les contourner.
    \item En mode préprogrammé : le robot s’arrête et signale un blocage.
\end{itemize}

\subsubsection{Détection de couleurs}
Le robot doit pouvoir reconnaître différentes zones colorées au sol (par exemple, repérer les zones blanches sur fond noir), en s’appuyant sur ses capteurs infrarouges ou sa caméra.

\subsubsection{Retour visuel et sonore}
Des retours visuels (\acrshort{led}s de différentes couleurs) et sonores doivent indiquer l’état du robot : par exemple, rouge pour un blocage, vert pour la réussite d’une tâche, etc.
