\subsection{Les serious games} \label{sec:serious_game}
L’apprentissage de la programmation et de la robotique éducative repose sur de nombreux outils et technologies visant à rendre ces disciplines plus accessibles.
Parmi les plus populaires, on retrouve \textit{Pybricks} (LEGO Mindstorms, Technic, ...), \textit{ThymioSuite}, \textit{Blockly} et \textit{Tinkercad}, qui facilitent la compréhension des concepts fondamentaux grâce à des interfaces visuelles intuitives voire ludiques. 

Dans l'enseignement supérieur, en particulier en Belgique, les \acrfull{vpl}, et plus particulièrement la \textit{programmation par blocs}, sont intégrés à des initiatives pédagogiques qui visent à éveiller l’intérêt des jeunes en favorisant la collaboration et l'apprentissage immersif.
À titre d’exemple, l’\textit{Initiation ludique à la programmation avec Minecraft} à l’Hénallux \autocite{christophe_leclere_initiation_2020}, et le \textit{Thymio Escape Game} \autocite{christian_giang_thymio_nodate} mis en place à l’UNamur lors du Printemps des Sciences 2025, illustrent cette approche.

Cette tendance en essor, souvent utilisée dans des contextes dits \textit{no-code} \autocites{noauthor_software_nodate}, simplifie la représentation des concepts complexes et rend l’apprentissage plus intuitif, notamment pour les débutants \autocites{noauthor_visual_2025}{batni_current_2025} car il exploite l’interactivité de l’interface pour réduire la distance perçue entre l’utilisateur et le programme.

Les \textit{serious games} (en français: jeux sérieux) désignent des jeux dont l'objectif dépasse ainsi le simple divertissement.
Ils tirent parti des codes et des mécaniques du jeu vidéo afin d’engager les utilisateurs dans une expérience immersive et motivante, dont le but est de transmettre des savoirs, développer des compétences, favoriser l’apprentissage ou influencer des comportements dans des contextes réels.

Selon Clark C. Abt, l’un des premiers à formaliser ce concept, un jeu sérieux est défini comme \blockquote{un jeu dont la finalité est explicitement éducative, bien que la forme demeure celle d’un jeu} \autocite{abt_serious_1970}.
Cette définition s’est élargie pour inclure toute application interactive exploitant les principes du jeu dans un but pédagogique, professionnel, sanitaire ou sociétal.

\subsubsection{Caractéristiques principales}
Les jeux sérieux se distinguent par plusieurs aspects fondamentaux:

\begin{itemize}
    \item \textbf{Une utilité clairement définie} : ces jeux sont conçus pour enseigner des connaissances, simuler des situations concrètes ou soutenir la prise de décision contextualisée \autocite{david_r_michael_serious_2006}.
    
    \item \textbf{Des mécaniques ludiques engageantes} : progression par niveaux, défis, récompenses et narration immersive sont employés pour favoriser la motivation de l’utilisateur \autocite{ute_ritterfeld_serious_2009}.
    
    \item \textbf{Un ancrage dans un domaine d’application spécifique} : les jeux sérieux sont largement utilisés en éducation \autocite{breuer_why_2010}, en formation professionnelle \autocite{r_realite_2021}, dans les milieux militaire et médical \autocite{axinesis_axinesis_nodate}, ou encore dans des démarches de sensibilisation à des enjeux sociaux \autocite{devresse_hunteroiinside-out_2023} ou environnementaux.
\end{itemize}

\subsubsection{Lien avec ce travail}
Dans le cadre de ce projet, qui s’adresse à un public d’élèves, il est pertinent de se situer dans la catégorie des jeux sérieux.
L’interface a pour vocation d’introduire de manière ludique à la robotique éducative, tout en favorisant l’intérêt pour ces technologies. 
Elle vise à stimuler la curiosité, l’exploration et l’apprentissage de concepts fondamentaux en informatique à travers une approche interactive.
