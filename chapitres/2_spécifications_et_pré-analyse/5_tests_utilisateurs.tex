\subsection{Tests utilisateurs}

Une fois le prototype applicatif développé, il faudra le mettre à l'essai par un groupe d'utilisateurs de la tranche d'âges ciblée.

\subsubsection{Phases de test}
Le suivi d'une session de test se déroule selon les phases suivantes:
\begin{itemize}
    \item Introduction:
        \begin{enumerate}
            \item Présentation brève des objectifs de la session ;
            \item Démonstration rapide des fonctionnalités du robot e-puck2 ;
        \end{enumerate}

    \item Prise en main:
        \begin{enumerate}
            \item Exercices simples: faire avancer/reculer le robot, le faire tourner, etc. ;
            \item Observation de la compréhension de l'interface par les testeurs ;
        \end{enumerate}

    \item Défis progressifs:
        \begin{enumerate}
            \item Suivre une ligne tracée au sol ;
            \item Éviter des obstacles placés aléatoirement au sol ;
            \item Réaliser un parcours avec des points de passage précis ;
            \item Bonus: détester des couleurs et réagir en conséquence.
        \end{enumerate}

    \item Évaluation: 
        \begin{enumerate}
            \item Questionnaire simple (e.g. qu'est-ce qui était facile/difficile?) ;
            \item Discussion libre (pour recueillir les impressions à chaud).
        \end{enumerate}
\end{itemize}

\subsubsection{Critères d’évaluation}
Les critères d'évaluation sont les suivantes:
\begin{enumerate}
    \item intuitivité de l'interface: les élèves peuvent-ils comprendre sans explications détaillées?
    \item autonomie: quel est le degré d'indépendance dans la résolution des défis?
    \item résolution des problèmes: quelle est la capacité à corriger leurs erreurs?
    \item engagement: comment sont l'intérêt et la motivation des élèves durant l'activité?
\end{enumerate}

\subsubsection{Analyse des résultats}
L'analyse des résultats se fera en plusieurs temps, en commençant par l'observation directe des comportements pendant l'activité puis en continuant ensuite sur l'analyse des réponses du questionnaire.
Enfin, des propositions d'améliorations basées sur les difficultés rencontrées peuvent être formulées pour ce travail.
