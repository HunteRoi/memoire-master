\subsection{Besoins fonctionnels} \label{sec:specs}

L’objectif d'une telle interface est de permettre la réalisation de trois grandes catégories d’actions, définies comme suit par le Dr. Tuci :  

\begin{itemize}
    \item \textbf{Navigation} : exécution de parcours pré-définis, comme la traversée d’un labyrinthe.
    
    \item \textbf{Exploration} : recherche et identification d’éléments spécifiques dans l’environnement, tels que des dalles de couleur.
    
    \item \textbf{Observation de comportements collectifs}\footnote{Comme indiqué en \autoref{sec:limites}, cette catégorie dépasse le cadre du présent projet, en raison des contraintes liées à la coordination multi-robots. Elle est néanmoins présentée ici afin d’offrir une vision d’ensemble et de justifier certains choix techniques facilitant son intégration future.} : mise en relation de plusieurs robots afin d’observer des interactions coordonnées, par exemple dans un scénario de type "jeu du chat et de la souris".
\end{itemize}

\subsubsection{Navigation}
Ce mode consiste à guider le robot selon un parcours précis, comme traverser un labyrinthe ou se déplacer dans une arène délimitée.
Les déplacements peuvent être générés de manière aléatoire ou suivre un itinéraire pré-défini.
Ce mode nécessite la maîtrise des systèmes de locomotion du robot : moteurs, roues et capteurs de proximité pour la détection d’obstacles.


\subsubsection{Exploration}
Dans ce mode, le robot doit rechercher des éléments visuellement identifiables, comme des zones blanches sur un sol noir.
Les trajectoires peuvent également être aléatoires ou définies à l’avance.
Ce mode implique des exigences matérielles plus importantes : capteurs infrarouges au sol, caméra embarquée, et traitement d’image pour l’identification des couleurs.


\subsubsection{Observation de comportements collectifs}
Ce mode vise à faire interagir plusieurs robots simultanément, par exemple à travers des scénarios collaboratifs ou compétitifs.
Un robot peut être programmé pour suivre ou fuir un autre, selon des règles simples.
Cela implique la gestion de la communication inter-robots et la coordination de leurs actions dans un espace partagé.


\subsubsection{Spécifications}
Afin de répondre à ces exigences, les éléments suivants constituent les spécifications fonctionnelles de l’interface à développer.

\paragraph{Communication en temps réel} 
Le robot doit pouvoir échanger des données en continu avec le poste de contrôle : réception des commandes et envoi des retours d’état.  
La notion de temps réel est ici définie comme la capacité à réagir dans un délai suffisamment court pour que l’interaction utilisateur-robot reste fluide et fiable.

Cela signifie évidemment que celui-ci doit être connecté à un réseau afin de communiquer avec le poste de contrôle.

\paragraph{Identification visuelle}
Chaque robot doit pouvoir être identifié facilement, notamment par le biais de la configuration personnalisable de ses \acrshort{led}s (couleurs distinctes entre les robots).

\paragraph{Configuration flexible des déplacements}
L’utilisateur doit pouvoir ajuster les déplacements selon plusieurs paramètres :
\begin{itemize}
    \item Trois niveaux de vitesse ;
    \item Une définition de la distance à parcourir ;
    \item Deux modes de déplacement : aléatoire ou pré-programmé.
\end{itemize}

\paragraph{Gestion des obstacles}
Le comportement face aux obstacles doit être adapté au mode de déplacement :
\begin{itemize}
    \item En mode aléatoire : le robot ajuste sa trajectoire pour les contourner.
    \item En mode préprogrammé : le robot s’arrête et signale un blocage.
\end{itemize}

\paragraph{Détection de couleurs}
Le robot doit pouvoir reconnaître différentes zones colorées au sol (par exemple, repérer les zones blanches sur fond noir), en s’appuyant sur ses capteurs infrarouges ou sa caméra.

\paragraph{Retour visuel et sonore}
Des retours visuels (\acrshort{led}s de différentes couleurs) et sonores doivent indiquer l’état du robot : par exemple, rouge pour un blocage, vert pour la réussite d’une tâche, etc.
