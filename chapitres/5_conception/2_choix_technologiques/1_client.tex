\subsubsection{Choix du framework pour le client lourd} \label{sec:client_code}

Dans ce projet, l’interface utilisateur est conçue comme une application de bureau, qui devra :
\begin{itemize}
    \item fonctionner sous Linux en priorité mais avec support pour Windows et macOS (multi-plateforme),
    \item accéder au système de fichiers local pour charger/sauvegarder des scripts ou des journaux d'exécution,
    \item offrir la possibilité d'évoluer vers des interactions riches (simulation, édition visuelle, gestion multi-robot).
\end{itemize}

Le \autoref{tab:comparison_clients} compare les options possibles selon des critères pertinents dans la \autoref{sec:choix_client}.
Le choix s’est porté sur \textbf{Electron}, car il nécessite moins de dépendances à installer et configurer, et qu'il offre :
\begin{itemize}
    \item une large compatibilité multi-plateforme,
    \item une grande souplesse en interface grâce aux technologies Web,
    \item un écosystème riche (au travers des dépendances NPM),
    \item une évolutivité naturelle vers des outils plus complexes (simulation, dashboards).
\end{itemize}

Bien que plus gourmand en ressources que Tauri, Electron garantit une meilleure maturité et une intégration plus facile avec les bibliothèques Web existantes (e.g. React-Flow pour la programmation par blocs).
