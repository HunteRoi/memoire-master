\subsubsection{Choix du langage pour le code embarqué sur le robot} \label{sec:robot_code}

Le robot e-puck2 utilisé dans ce projet est équipé d’un Raspberry Pi Zero W tournant sous Linux.
Deux langages sont disponibles nativement sur l’e‑puck2 :
\begin{itemize}
  \item \textbf{Python} : déjà installé, facile à utiliser, très répandu en robotique éducative.
  \item \textbf{C++} : compilable via \textit{gcc} présent sur le robot, plus performant mais aussi plus complexe.
\end{itemize}

Le \autoref{tab:comparison_robot_languages}, disponible en \autoref{sec:annexes_a}, compare les options possibles selon des critères pertinents.
Les critères et l'échelle d'évaluation y sont également disponibles.
Ci-dessous, le \autoref{tab:robot_languages_scores_color} donne un résumé avec uniquement la pondération de chaque langage pour chaque critère et le score final.

\begin{longtable}[H]{|p{0.30\textwidth}|c|c|}
\caption{\label{tab:robot_languages_scores_color} Pondération des langages embarqués} \\

\hline
\textbf{Critère} & \textbf{Python} & \textbf{C++} \\
\hline
\endfirsthead

\multicolumn{3}{c}{\textit{Suite du tableau \ref{tab:robot_languages_scores_color}}} \\
\hline
\textbf{Critère} & \textbf{Python} & \textbf{C++} \\
\hline
\endhead

Déjà installé sur le robot       & \cellcolor{green!20}5 & \cellcolor{green!20}5 \\
Facilité d’écriture              & \cellcolor{green!20}5 & \cellcolor{yellow!30}3 \\
Performance                      & \cellcolor{yellow!30}3 & \cellcolor{green!20}5 \\
Consommation mémoire             & \cellcolor{yellow!30}3 & \cellcolor{green!20}5 \\
Support robotique                & \cellcolor{green!20}5 & \cellcolor{green!20}5 \\
Courbe d’apprentissage           & \cellcolor{green!20}5 & \cellcolor{yellow!30}3 \\
Modernité                        & \cellcolor{green!20}5 & \cellcolor{blue!15}4 \\
\hline
\textbf{Total (/35)}             & \cellcolor{green!25}\textbf{31} & \cellcolor{green!20}\textbf{30} \\
\hline
\textbf{Score \%}                & \cellcolor{green!25}\textbf{89\%} & \cellcolor{green!20}\textbf{86\%} \\
\hline

\end{longtable}

Le choix s’oriente donc vers \textbf{Python}.
Il est déjà installé et donc supporté par le Raspberry Pi Zero W, est aussi facile à lire et écrire pour des élèves, compatible avec de nombreux modules utiles et performant.
